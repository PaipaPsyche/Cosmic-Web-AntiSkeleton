\documentclass[12pt]{article}

\usepackage{graphicx}
\usepackage{epstopdf}

%\usepackage[english]{babel}
\usepackage[utf8]{inputenc}
\usepackage[spanish]{babel}
\usepackage{amsmath}
\usepackage{fancyhdr}
\usepackage[T1]{fontenc} 

\usepackage{hyperref}
\usepackage[left=3cm,top=3cm,right=3cm,nohead,nofoot]{geometry}
\usepackage{braket}
\usepackage{datenumber}
%\newdate{date}{10}{05}{2013}
%\date{\displaydate{date}}

\begin{document}

\begin{center}
\Huge
Caracterización de supercúmulos de galaxias en simulaciones

\vspace{3mm}
\Large David Leonardo Paipa León


\large
201516988


\vspace{2mm}
\Large
Director: Jaime Ernesto Forero

\normalsize
\vspace{2mm}

\today
\end{center}


\normalsize
\section{Introducción}

%Introducci�n a la propuesta de Monograf�a. Debe incluir un breve resumen del estado del arte del problema a tratar. Tambi�n deben aparecer citadas todas las referencias de la bibliograf�a (a menos de que se citen m�s adelante, en los objetivos o metodolog�a, por ejemplo)

En 1932 Harlow Shapley y Adelaide Ames publican un catálogo de
galaxias cercanas y sus respectivas distribuciones en el
espacio\cite{catalog}\cite{Shapley}.
Este trabajo rebel\'o agrupaciones de galaxias a gran escala. 
Hasta entonces se tenía conocimiento de galaxias agrupadas en grupos
pequeños, llamados cúmulos ,  pero las observaciones sugerían que estos formaban parte de estructuras
compuestas de varios de estos cúmulos. 
Casi un siglo despu\'es observadores lograron definir un 
el supercúmulo local dentro del cual se incluye la Vía Láctea \cite{tully}. 
Este supercúmulo se define como una red de estructuras (formadas por galaxias)
que tienen velocidades convergentes hacia un gran atractor.
Este superc\'umlo recibe el nombre Laniakea y cuenta con un radio
aproximado de 160 Mpc y la masa de cerca de $10^{17}$ soles
\cite{nature}.

A pesar de que la astronomía moderna ha avanzado bastante en el
estudio de cúmulos de galaxias, aún se desconoce mucho sobre las
propiedades estadísticas de estos supercúmulos. 
Para resolver
algunas incognitas en este tópico es necesario utilizar simulaciones.

El objetivo de esta tesis es encontrar y caracterizar supercúmulos en
simulaciones cosmol\'ogicas.
Para hallar los supercúmulos buscaremos regiones en donde l\'ineas de
flujo de galaxias converjan.
Buscamos segmentar el volumen completo de la simulaci\'on para
asignarle pertenencia a cada galaxia simulada a un superc\'umulo
determinado. 
Este trabajo es una continuaci\'on directa del trabajo de Sergio
Hern\'andez quien implement\'o un algoritmo muy aproximado para 
encontrar superc\'umulos \cite{Hernandez}. 
La limitaci\'on de ese trabajo es que no era una segmentaci\'on del
volumen total y subestimaba el volumen y masa de los superc\'umulos.
La novedad de nuestra tesis es que implementaremos un seguimiento
detallado de los flujos de velocidades para lograr un mayor realismo
con respecto a los procedimientos utilizados para encontrar a Laniakea
en las observaciones.


\section{Objetivo General}

%Objetivo general del trabajo. Empieza con un verbo en infinitivo.

Crear un algoritmo para encontrar supercúmulos en simulaciones de universo a gran escala.


\section{Objetivos Específicos}

%Objetivos espec�ficos del trabajo. Empiezan con un verbo en infinitivo.

\begin{itemize}
	\item Implementar un algoritmo en C que pueda buscar supercúmulos en simulaciones.
	\item Aplicar el algoritmo a simulaciones cosmológicas.
	\item Caracterizar los supercúmulos encontrados con el algoritmo.
\end{itemize}

\section{Metodología}

%Exponer DETALLADAMENTE la metodolog�a que se usar� en la Monograf�a. 
%Monograf�a te�rica o computacional: �C�mo se har�n los c�lculos te�ricos? �C�mo se har�n las simulaciones? �Qu� requerimientos computacionales se necesitan? �Qu� espacios f�sicos o virtuales se van a utilizar?
\subsection{Implementación teórica en el algoritmo}
Dada una simulación de N-cuerpos con datos de posición, velocidad y
masa en 3 Dimensiones, se dividirá el espacio en una grilla con el fin
de hallar la velocidad de cada voxel . El cálculo de la velocidad del
centro de masa para cada voxel viene dada por: 
\begin{equation}
V_{CM}=\frac{\sum_{i=1}^n m_i \textbf{v}_i}{\sum_{i=1}^n m_i} 
\end{equation}
Donde \textit{n} es el número de elementos dentro del voxel, $m_i$ la masa del elemento i y $v_i$ la velocidad de este.
Así se obtiene un nuevo mapa vectorial en el espacio con las velocidades del centro de masa, que son más fáciles de tratar. Ahora, Para construir el mapa escalar de \textit{densidad de flujo de velocidad} a cada voxel en la posición $(i,j,k)$ se le asigna un escalar de fujo:
\begin{equation}
\Phi_{i,j,k}=Vx_{i-1,j,k}-Vx_{i+1,j,k}+Vy_{i,j-1,k}-Vy_{i,j+1,k}+Vz_{i,j,k-1}-Vz_{i,j,k+1}
\end{equation}
Donde $Vx$ , $Vy$ y $Vz$ son las velocidades en X, Y y Z respectivamente de determinado voxel. Notar que este coeficiente  es mayor si las velocidades de los vecinos inmediatos del pixel en cuestion son entrantes. Después, se usa el algoritmo de Watershed \cite{WaterBeuch} \cite{SegmBeuch} con el fin de determinar los máximos o mínimos locales en donde el material está \textit{entrando} o \textit{saliendo} respectivamente, como lo hacen Sousbie et. al. \cite{FullyConn}. Se van a reconocer todos los pixeles que pertenecen a un mismo supercúmulos haciendo un mapeo iterativo a partir del punto crítico, primiero mirando los vecinos inmediatos al punto crítico y depués los vecinos inmediatos de esos pixeles, para asignar a esa región los pixeles con flujos que eventualmente convergeran a este punto.


\subsection{Manejo de datos y Herramientas}

%Se usan datos de simulaciones de N-Cuerpos \cite{simulations} que estan disponibles públicamente .Para hacer pruebas iniciales de funcionamiento de algoritmos y visualización de datos en 2 dimensiones actualmente se usa Python, esto con el fin de hacer algoritmos de prueba que sean de ayuda en la creación del algoritmo principal\footnote{Se comparan resultados con los obtenidos por Novikov et. al. \cite{probe2d} o Sousbie\cite{Persistent} en algoritmos 2D}. Eventualmente para la monografía se pasará el desarrollo del algoritmo a C para tratar cantidades de datos más grandes y se seguirá usando Python con el fin de visualizar datos, pero no para procesarlos.
El desarrollo de la monografía puede ser llevado a cabo desde un
computador portátil, pues la complejidad de los algoritmos no requiere
una gran cantidad de procesamiento.Los datos que se analizarán
provienen de simulaciones públicas\cite{simulations}. 




\section{Consideraciones \'Eticas}

Puesto que la monografía va a ser enteramente computacional, no es
necesario que el proyecto sea estudiado por el comité de ética de la
Facultad de Ciencias. 
Seguiremos las pr\'acticas adecuadas al citar el trabajo o resultados
de otras personas que utilicemos en nuestro trabaj. 

%A partir del periodo 2017-20 debe incluirse en el formato de propuesta de monografía una sección titulada Consideraciones éticas. Esta sección debe incluir los detalles relacionados con aspectos éticos involucrados en el proyecto. Por ejemplo, se puede describir el protocolo establecido para el manejo de datos de manera que se asegure que no habrá manipulación de la información, ni habrá plagio de los mismos. También se puede tener en cuenta si hay algún conflicto de intereses involucrado en el desarrollo del proyecto o se puede detallar si el trabajo está relacionado con las actividades y poblaciones humanas mencionadas en el siguiente link https://ciencias.uniandes.edu.co/investigacion/comite-de-etica. Es importante tener en cuenta que esta sección debe incluir una frase explícita sobre si el proyecto debe pasar o no a estudio del comité de ética de la Facultad de Ciencias.


\section{Cronograma}

\begin{table}[htb]
	\begin{tabular}{|c|cccccccccccccccc| }
	\hline
	Tareas $\backslash$ Semanas & 1 & 2 & 3 & 4 & 5 & 6 & 7 & 8 & 9 & 10 & 11 & 12 & 13 & 14 & 15 & 16  \\
	\hline
	1 & X & X &   &    &   &   &   &   &   &   &   &   &   &   &   &   \\
	2 &   &   & X &  X & X &   &   &   &   &   &   &   &   &   &   &   \\
	3 &   &   &   &    & X &  X &  &  &   &   &   &   &   &   &   &   \\
	4 &   &   &   &    &   &   X & X & X & X & X &   &   &   &   &   &   \\
    5 &   &   &   &    &   &     &   & X &   &   &   &   &   &   &   &   \\
	6 &   &   &   &   &  &   &   &   &  &   &  X & X &   &   &  &   \\
    7 & X  &  X &   &   &  &   &   &   &   &   &   &  X&  X &  X &  X&   \\
	\hline
	\end{tabular}
\end{table}
\vspace{1mm}

\begin{itemize}
	\item Tarea 1: Implementar un algoritmo de prueba en python para segregar  supercúmulos en 3D.
	\item Tarea 2: Crear códigos de prueba en C para leer, analizar y segmentar grandes cantidades de datos de simulaciones.
    
	\item Tarea 3: Implementar un algorimo de búsqueda de puntos críticos de densidad en simulaciones 3D en C.
	\item Tarea 4: Hacer un algoritmo en C capaz de segregar supercúmulos en 3D.
    \item Tarea 5: Presentar aavances correspondientes al 30\%
    \item Tarea 6: Caracterizar los supercúmulos.
    \item Tarea 7: Escribir el documento de monografía.
    
    
\end{itemize}

\section{Personas Conocedoras del Tema}

%Nombres de por lo menos 3 profesores que conozcan del tema. Uno de ellos debe ser profesor de planta de la Universidad de los Andes.

\begin{itemize}
	\item Jaime Ernesto Forero Romero (Universidad de los Andes)
    \item Marcela Hern\'andez (Universidad de los Andes)
	\item Benjamin Oostra Van Noppen (Universidad de los Andes)
	\item Alejandro García Varela (Universidad de los Andes)
\end{itemize}


\begin{thebibliography}{10}

%\bibitem{Jerry} J. Banks. \textit{Discrete-Event System Simulation}. Fourth Edition. Prentice Hall International Series in Industrial and Systems Engineering, pg 86 - 116 y 219 - 235, (2005).
\bibitem{catalog} A.Sandage.,G. Tammann\textit{A revised Shapley-Ames catalog}, (1981).

\bibitem{Shapley} G. de Vauncouleurs.\textit{Is the local supercluster a random clumping accident}, \textbf{1}(1976).
\bibitem{tully} Tully, R. B., Courtois, H., Hoffman, Y., \& Pomarede, D, 2014, Nature, 513, 71
\bibitem{nature} E. Gibney, \textit{Earth's new address: 'Solar System, Milky Way, Laniakea'}, Nature (2014).

%\bibitem{Hernandez} S. Hernandez.\textit{Laniakea in a Cosmological Context or Detection of galaxy superclusters in simulated cosmological structures}, pg 5 , (2016).



\bibitem{WaterBeuch} S. Beucher \& C. Lantuejoul. \textit{Use of Watersheds in contour detection}.(1979) 
\bibitem{SegmBeuch} S. Beucher. \textit{The watershed transformation applied to image segmentation}.

\bibitem{FullyConn} T. Sousbie, S. Colombi \& C. Pichon. \textit{The fully connected N dimensional Skeleton: probing the evolution of the cosmic web}, \textbf{1}(2018).

\bibitem{simulations} Índice de SUSimulations . Simulaciones obtenidas de \textit{ftp://skun.iaa.es/SUsimulations/C1.2/HALOLISTS/}, Consultado el 24 de Abril de 2018.


\bibitem{probe2d} D. Novikov, S. Colombi \& O. Doré. \textit{Skeleton as a probe of the cosmic web: the two-dimensional case}, \textbf{1}(2005).

\bibitem{Persistent} T. Sousbie. \textit{The persistent cosmic web and its filamentary structure I: Theory and implementation}, \textbf{1}(2010).







%\bibitem{LabInt} P. D�az \& N. Barbosa: \textit{Obtenci�n de n�meros aleatorios}. Informe final del curso Laboratorio Intermedio. Universidad de Los Andes, Bogot�, Colombia, (2012).



\end{thebibliography}

\section*{Firma del Director}
\vspace{1.5cm}

\section*{Firma del Codirector	}





\end{document} 
